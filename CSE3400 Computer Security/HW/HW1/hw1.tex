w\documentclass[12pt]{article}
\usepackage{amsmath,amssymb,epsfig,boxedminipage,helvet,theorem,endnotes,version}
\usepackage{mathtools}
\newcommand{\remove}[1]{}
\setlength{\oddsidemargin}{-.2in}
\setlength{\evensidemargin}{-.2in}
\setlength{\textwidth}{6.5in}
\setlength{\topmargin}{-0.8in}
\setlength{\textheight}{9.5in}


\newtheorem{theorem}{Theorem} 
\newtheorem{definition}{Definition}

\newcommand{\gen}{\mbox{\tt Gen}}
\newcommand{\enc}{\mbox{\tt Enc}}
\newcommand{\dec}{\mbox{\tt Dec}}

\newcommand{\sol}{{\bf Solution}: }

\newcommand{\zo}{\{0,1\}}
\newcommand{\zoell}{\{0,1\}^\ell}
\newcommand{\zon}{\{0,1\}^n}
\newcommand{\zok}{\{0,1\}^k}

%% eg \priv{eav}{n,b} for def 3.9 style, or \priv{eav}{n} for def 3.8
\newcommand{\priv}[2]{${\tt Priv}_{A,\Pi}^{\tt #1}(#2)$}

\newcommand*\concat{\mathbin{\|}}

\newcommand\xor{\oplus}

\newcommand{\handout}[2]{
\renewcommand{\thepage}{\footnotesize CSE 3400/CSE 5850, #1, p. \arabic{page}}
\begin{center}

\noindent
{\bf UConn, School of Computing}

\noindent
{\bf Fall 2024}

\noindent
{\bf CSE 3400/CSE 5080: Introduction to Computer and Network Security \\/ Introduction to Cybersecurity}
\end{center}

\begin{center}
{\Large #1}
\end{center}
}



\begin{document}

\handout{Assignment 1}{}

\noindent
{Instructor: Prof.~Ghada Almashaqbeh}

\noindent
{Posted: 9/9/2024}

\noindent
{Submission deadline: 9/17/2024, 11:59 pm} \\\\


\noindent{\bf Problem 1 [30 points]\\}
Write a program that increments a counter $2^{10}, 2^{20}$, and $2^{30}$ times (so the initial counter value will be 0, then 1, 2, ..., up to $2^{10}$, then repeat with 0, 1, ...., $2^{20}$, and so on). 
\begin{enumerate}
\item For each counter value, in each iteration compute the bitwise AND of the counter with the value 1254632 (so let the counter be $ctr = 0$, compute $ctr \; \& \; 1254632$, then set $ctr = 1$ and compute that value again, and so on). Do that for $2^{10}, 2^{20}$, and $2^{30}$ counter values. Measure how many seconds your program takes to run in each case (list the machine specifications and the programming language you have used). Estimate how many years your program would take to do this operation when the counter is incremented $2^{480}$ times.

\item Repeat the previous counter increments but now in each iteration compute the following formula: $ctr \oplus 990 +  ctr/3456$ (so there is a bitwise XOR and a division operation). Measure how many seconds your program takes to run in each case (list the machine specifications and the programming language you have used). Estimate how many years your program would take to do this operation when the counter is incremented $2^{480}$ times. 
\end{enumerate}


% ***********************************************************************************************

\noindent{\bf Problem 2 [40 points]\\}
Let $G_0:\zo^n \to \zo^{2n}$ and $G_1:\zo^{n/2} \to \zo^{n}$ be PRGs. For each of the following constructions, prove whether it is a PRG or not. To prove insecurity you have to show an attack against the scheme and provide an informal argument about the attack success probability. For security provide a convincing informal argument why it is a PRG (with computing the probabilities as needed).
\begin{enumerate}
\item $G_2(s \concat t)= G_0(s) \concat (t \xor 1^n)$ 

\item $G_3(s \concat z)= G_0(s \oplus z)$

\item $G_4(s \concat z) =  \mathsf{LH}(G_0(s)) \oplus G_1(\mathsf{RH}(z))$, where $\mathsf{LH}$ is the left half of the input string, and $\mathsf{RH}$ is the right half of the input string.

\item $G_5(s) = (G_0(s) \;\; \mathsf{mod} \;\; 2) \concat G_0(s)$, where $\mathsf{mod}$ is the modulus operation.
\end{enumerate} 
($\concat$ is the concatenation operation, $\oplus$ is the bitwise XOR operation, $s$ and $z$ are random strings each of which is of length $n$ bits, and $t$ is any string---not necessarily random---of length $n$ bits.)\\


% ***********************************************************************************************
\newpage
\noindent{\bf Problem 3 [30 points]\\}
\noindent\emph{Part 1:} Answer the following for the monoalphabetic substitution cipher.
\begin{enumerate}
\item Assume you have a CTO attacker who is trying to perform the letter frequency attack we studied in class. Would the attacker be more successful in retrieving the key (the key is basically the permutation of the alphabet) if he gets a long ciphertext or a short one? why? 

\item Will your answer change for the previous question if we consider a CPA attacker? why?
\end{enumerate}


\noindent\emph{Part 2:} Answer the following about OTP (just provide informal convincing arguments, no need for formal proofs).
\begin{enumerate}
\item Alice encrypted two messages of length $n$ bits to Sponge using pads $p_1$ and $p_2$, respectively (producing two ciphertexts $c_1$ and $c_2$, respectively). She also posted the value $y_1 = p_1 \oplus x$ and $y_2 = p_2 \oplus x$ on her website and argued that these masked values are also random and will not reveal anything about the pads (and she did not tell anyone that value of $x$, however, everyone knows that the posted valued are the XOR of the pads with the same unknown value $x$). Is the scheme above secure against a CTO attacker? why?

\item Eve wants to send two messages of length $n$ bits to Charlie. She generated a random pad $p$ of length $n$ bits and sent two ciphertexts as follows: $c_1 = p \oplus m_1$ and $c_2 = (p+1) \oplus m_2$, and she argues that this is secure. Is this secure against a CTO attacker? why?
\end{enumerate}


% ***********************************************************************************************

\end{document} 